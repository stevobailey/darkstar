\title{Exploring Digital Sampling, Fourier Transforms, and Sideband Mixers}
\date{\today}
\author{Stevo Bailey}

\documentclass[12pt]{article}

\usepackage[english]{babel}
\usepackage[utf8x]{inputenc}
\usepackage{amsmath}
\usepackage{graphicx}
\usepackage{titling}
\usepackage{caption}
\usepackage{subcaption}
%\usepackage{hyperref}
\usepackage{color}
\usepackage[normalem]{ulem}
\usepackage{titling}

\pretitle{\begin{center}\Huge\bfseries}
\posttitle{\par\end{center}\vskip 0.5em}
\preauthor{\begin{center}\Large}
\postauthor{\end{center}}
\predate{\par\large\centering}
\postdate{\par}


\usepackage{epstopdf}
\graphicspath{{./figures/}}
\DeclareGraphicsExtensions{.eps,.png}


\newcommand{\degree}{\ensuremath{^\circ} }

\begin{document}
\maketitle

\section{Introduction}
The first radio astronomy lab introduces a number of concepts, including the hardware side of radio signal acquisition, sampling theory and the Nyquist criterion, Fourier analysis and the Fourier transform, mixing, IDL, and \LaTeX.
This lab report covers most of these goals, with experience in \LaTeX gained from simply writing the report.
The goals were to familiarize the students with the concepts through hands-on analysis in the radio astronomy lab.
Frequency generators, an oscilloscope, and a PicoScope were used to generate and observe sinusoidal signals.
Power splitters and ZAD-1 mixers allowed for real-time signal modulation.
Once data were generated, IDL was used to collect the data, post-process it, and plot it.
The post-processing included testing the Nyquist criterion, exploring the DFT and FFT, comparing time series data and frequency data, and performing Fourier filtering.
Finally, the results were summarized in this report. 



\section{Nyquist Sampling and Aliasing}
The goal of the first experiment was to explore the Nyquist criterion.
The Nyquist criterion states that when digitally sampling a signal, perfect reconstruction is possible by sampling at greater than or equal to twice the maximum frequency present in the signal.
Sampling faster gives no extra information, and thus wastes space.
Sampling slower does not guarantee perfect reconstruction, and aliasing effects occur.
The following experiment illustrates this.

We used a PicoScope to digitally sample the output of a function generator at 6.25 MHz, also known as $\nu_{smpl}$.
We set the function generator to produce signals at integer tenths of the sampling frequency, so 
\begin{equation}
\nu_{sig} = (0.1, 0.2, 0.3, ..., 0.9, 1.0)\nu_{smpl}.
\end{equation}
Finally, we took one more digital sample with the signal frequency much bigger than the sampling frequency, specifically 82 times larger.
According to the Nyquist criterion, we expected to see correct spectra for $\nu_{sig} \leq 0.5\nu_{smpl}$.
Otherwise, the signal was under-sampled, so we expected to see aliasing frequencies.

The aliasing effect can even be seen in the time domain.
Figure~\ref{fig:nyquist_time1} shows time-domain plots for a subset of the sampled frequencies.
Aliasing produces frequencies which fold back away from the Nyquist frequency, meaning $\nu_{sig}=0.4\nu_{smpl}$ has the same frequency components as $\nu_{sig}=0.6\nu_{smpl}$.
Time-domain signals with similar expected frequency spectra are plotted next to each to show this effect.
Note when the sampled signal is exactly half the sampling frequency, the same two points are sampled repeatedly, as expected.

\begin{figure}
\centering
\includegraphics[width=0.95\linewidth]{nyquist_time1}
\caption{Time-domain plots of signals at integer tenths of the sampling frequency.}
\label{fig:nyquist_time1}
\end{figure}

The frequency spectra, obtained by the DFT function in IDL, show this aliasing more concretely.
Figure~\ref{fig:nyquist_freq1} again shows the signals paired up the same way as in Figure~\ref{fig:nyquist_time1}.
The largest frequency components in each pair match.
When $\nu_{sig}=0.5\nu_{smpl}$, the dominant frequency lies directly at the maximum frequency given by the DFT.
The final two sampled signals are shown in Figure~\ref{fig:nyquist_freq2}.
When $\nu_{sig}=\nu_{smpl}$, the dominant frequency is zero, since we sample the same value within a period each time.
When $\nu_{sig}\gg\nu_{smpl}$, aliasing has folded back the dominant frequency to some value within the sampling range.


\begin{figure}
\centering
\includegraphics[width=0.95\linewidth]{nyquist_freq1}
\caption{Frequency-domain plots of signals at integer tenths of the sampling frequency.}
\label{fig:nyquist_freq1}
\end{figure}

\begin{figure}
\centering
\includegraphics[width=0.45\linewidth]{nyquist_freq2}
\caption{Frequency-domain plots of the other two sampled signals.}
\label{fig:nyquist_freq2}
\end{figure}

\section{DFT vs. FFT}

Previous spectra were calculated using the DFT, which takes quite some time for large datasets.
For each frequency point, the DFT integrates (or sums) all datapoints.
Since there are N frequencies and N datapoints, the computation goes as $O(N^2)$.
However, the fast Fourier transform (FFT) is exactly the same as the DFT, but takes advantage of redundancy and sparsity within the algorithm.
Thus it achieves $O(N\log_2 N)$ complexity, saving significant computation time.
When sampling N channels, the FFT produces N outputs at frequencies $0, \nu_{smpl}/N, 2\nu_{smpl}/N, ..., \nu_{smpl}$.
Above $\nu_{smpl}/2$, the frequencies are really aliases, so $\nu_{smpl}/2+\epsilon=-\nu_{smpl}/2+\epsilon$, where $\epsilon$ is some value between 0 and $\nu_{smpl/2}$.

\section{Spectral Leakage}
Spectral leakage is a phenomenon that occurs because sampling occurs for a finite amount of time.
This creates a window function, which when passed through a Fourier transform, produces spectral components not present in the signal. 
To show this, we performed the DFT using smaller spectral bins than the usual $\nu_{smpl}/N$.
Figure~\ref{fig:leakage} shows that lots of higher peaks appear when smaller spectral bins are used.
In particular, near the origin, the terms around DC are much higher when the resolution is increased.
This shows the spectral leakage, which can mask interesting data in that region.

\begin{figure}
\centering
\includegraphics[width=0.45\linewidth]{leakage}
\caption{Spectral leakage is evident by increasing the frequency resolution}
\label{fig:leakage}
\end{figure}


\section{Double-Sideband Mixers}
Mixers down-convert or up-convert signals from one frequency to another.
For example, audio signals are up-converted from their nominal range (20 Hz to 20 kHz) to radio frequencies (hundreds of kHz) before broadcasting.
High-frequency signals may be down-converted to lower frequencies for simpler processing and filtering.
To accomplish this, mixers simply multiply the signal of interest (often called the RF) with some local sinusoid (called the local oscillator (LO) or intermediate frequency (IF)).
This produces a new signal containing the sum and difference frequencies.
Often this output is filtered and further processed for its particular interest.

To explore double-sideband (DSB) mixers, we applied two signals similar in frequency to a ZAD-1 balanced mixer.
We did this twice.
First, we used $\nu_{LO}=$7 MHz and $\nu_{RF}=$7.2 MHz.
We also tried $\nu_{LO}=$7 MHz and $\nu_{RF}=$6.8 MHz.
Each time we sampled at 31.25 MHz.
Power spectra of these two cases are shown in Figure~\ref{fig:dsb_power}.
Here $\delta\nu$ is 0.2 MHz.
The peak frequencies appear to be in the right places, but to make sure, Figures~\ref{fig:dsb_power_closeup1} and \ref{fig:dsb_power_closeup2} show the sum and difference frequencies up close.
As expected, the 14.2 MHz peak is higher when $\nu_{RF}=$7.2 MHz, and the 13.8 MHz peak is higher when $\nu_{RF}=$6.8 MHz.
The other peak in each one is also quite high, but this is due to aliasing.

\begin{figure}[h]
    \centering
    \begin{subfigure}[b]{0.3\textwidth}
        \includegraphics[width=1.09\linewidth]{dsb_power}
        \caption{Full spectra}
        \label{fig:dsb_power}
    \end{subfigure}
    \quad
    \begin{subfigure}[b]{0.3\textwidth}
        \includegraphics[width=\linewidth]{dsb_power_closeup1}
        \caption{Close-up of DC}
        \label{fig:dsb_power_closeup1}
    \end{subfigure}
    \begin{subfigure}[b]{0.3\textwidth}
        \includegraphics[width=1.01\linewidth]{dsb_power_closeup2}
        \caption{Close-up of 14 MHz}
        \label{fig:dsb_power_closeup2}
    \end{subfigure}
    \caption{Power spectra for two RF frequencies.}
\end{figure}

As mentioned earlier, it is sometimes useful to filter the output signal and obtain only the sum or difference component.
To demonstrate this, we used Fourier filtering, which involves setting desired spectral frequencies to zero, then taking the inverse Fourier transform to recreate the signal.
We used a low-pass Fourier filtering technique to obtain just the difference signal.
The result, in Figure~\ref{fig:filter}, show that the high-frequency component of the time-domain signal clearly disappears, as expected.
The low-frequency component looks the same as before.

\begin{figure}
\centering
\includegraphics[width=0.95\linewidth]{filter}
\caption{Demonstrating the Fourier low-pass filtering technique.}
\label{fig:filter}
\end{figure}

\section{Sideband-Separating Mixers}
All previous DSB plots show perfect frequency-domain symmetry around the origin.
This results from taking the Fourier transform of a purely real signal.
However, these symmetric sidebands are really some combination of the RF signals above and below the LO sidebands. 
Ideally we would like to distinguish the sidebands.
We can accomplish this by introducing an imaginary signal into the Fourier transform, which consists of a 90\degree phase shift of the LO.
A 90\degree shift corresponds to a delay of $\lambda/4$, which we achieved by using a wire of appropriate length.
We verified the phase shift by measuring the signal offset on the oscilloscope.

For this experiment we chose frequencies $\nu_{LO}=$10.5~MHz and $\nu_{RF}=$10.1~MHz and 10.9~MHz, and $\nu_{smpl}=$62.5~MHz.
The delayed signal was used as the imaginary values of the complex FFT input signal.
Figure~\ref{fig:ssbshort} shows that no phase delay produces the typical double-sideband mixer result.
Each sideband looks equivalent.
However, as seen in Figure~\ref{fig:ssb}, adding a 90\degree phase delay allows us to distinguish the signal sidebands.
This is most evident around the origin, where the difference frequency appears.
When $\delta\nu$ is positive (the top plot in Figure~\ref{fig:ssb}), the +4~MHz frequency has more power.
But when $\delta\nu$ is negative (the bottom plot in Figure~\ref{fig:ssb}), the -4~MHz frequency has more power.

\begin{figure}
\centering
\includegraphics[width=0.85\linewidth]{ssbshortplots}
\caption{Results of a sideband-separating mixer when no phase delay is added}
\label{fig:ssbshort}
\end{figure}

\begin{figure}
\centering
\includegraphics[width=0.85\linewidth]{ssbplots}
\caption{Results of a sideband-separating mixer when a 90\degree phase delay is added}
\label{fig:ssb}
\end{figure}

\section{Conclusion}
This report explored a range of concepts.
These include the Nyquist criterion, sampling theory, and mixing for frequency conversion.
Sampling a signal at a frequency less than the maximum signal frequency produces imperfect digital reconstruction and aliasing.
The DFT and FFT perform exactly the same computation for powers-of-two datapoints, but the FFT reduces computational complexity significantly.
Digital sampling in general produces spectral leakage (aliasing), and digital sampling over a finite time frame produces undesirable side lobes (spectral leakage).
Frequency conversion is useful for up-converting or down-converting a signal to some desired frequency, but it must be done with care.
The two sidebands tend to merge together unless a sideband-separating mixer and FFT are used.
These well-known results were explored in this report, but in the future, it will be important to remember these concepts when exploring actual astronomy data.


\enddocument
