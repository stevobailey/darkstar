\title{Radio Interferometry at X Band}
\date{\today}
\author{Stevo Bailey}

\documentclass[12pt]{article}

\usepackage[english]{babel}
\usepackage[utf8x]{inputenc}
\usepackage{amsmath}
\usepackage{graphicx}
\usepackage{titling}
\usepackage{caption}
\usepackage{subcaption}
%\usepackage{hyperref}
\usepackage{color}
\usepackage[normalem]{ulem}
\usepackage{titling}

\pretitle{\begin{center}\Huge\bfseries}
\posttitle{\par\end{center}\vskip 0.5em}
\preauthor{\begin{center}\Large}
\postauthor{\end{center}}
\predate{\par\large\centering}
\postdate{\par}


\usepackage{epstopdf}
\graphicspath{{./figures/}}
\DeclareGraphicsExtensions{.eps,.png,.pdf}


\newcommand{\degree}{\ensuremath{^\circ} }

\begin{document}
\maketitle

\section{Introduction}
This lab introduces interferometry as a way of measuring declinations of point sources and widths of extended sources.
Interferometry uses multiple dishes to produce an interference pattern, or ``giant sine wave in the sky".
The sine wave's properties depend on the declination, signal wavelength, array baseline, object brightness, and time (hour angle).
Using interferometry measurements and least-squares fitting, accurate source properties can be determined.



\section{Measurement Setup}
Before taking interferometry data, a script was written to track an object in the sky.
The script either uses the IDL procedures \texttt{isun} or \texttt{imoon} to calculate an altitude and azimuth, or it manually calculates it from known right ascension and declination values.
Right ascension and declination values come from the J2000 equinox, which is a consistent set of benchmarks pegged to the year 2000 for a number of sources.
These values are precessed to the current equinox, then converted to local altitude and azimuth values using the local sidereal time (LST) and Campbell Hall's latitude and longitude.
The script updates the values every 10 seconds, using the \texttt{point2} command to set the dish orientations.
It also re-calibrates the dishes about every hour using the \texttt{homer} command, in case the dishes lose alignment.

The script was tested on the Sun.
Verification was done by both visual confirmation and data collection.
To visually confirm alignment, it was noted that the feed shadow should line up with the center of the dish.
In reality, it was close, but not exactly on the center.
The shadows for each dish were approximately three centimeters away from the center at most.
Assuming the feed is one meter from the dish, this means the dishes are about 1.7\degree off from the desired position.
For extended sources, this offset reduces the fringe amplitude but otherwise has no large effect.
For point sources, this offset reduces the chances of successfully pointing at the object, possibly preventing any measurement of the object.
Also, it was noted that wind atop Campbell Hall shakes the dishes, moving them by several degrees or more.


\section{Point Source}
\section{Sun and Moon}

\subsection{Signal Quantization}

%% subfigure:
%\begin{figure}[h]
%    \centering
%    \begin{subfigure}[b]{0.44\textwidth}
%        \includegraphics[width=1.09\linewidth]{1v_hist}
%        \caption{Max voltage of 1 V}
%        \label{fig:1vhist}
%    \end{subfigure}
%    \quad
%    \begin{subfigure}[b]{0.46\textwidth}
%        \includegraphics[width=\linewidth]{100mv_hist}
%        \caption{Max voltage of 100 mV}
%        \label{fig:100mvhist}
%    \end{subfigure}
%    \caption{Voltage histograms showing quantization extent}
%\end{figure}
%
%% Figure:
%\begin{figure}
%\centering
%\includegraphics[width=0.75\linewidth]{raw_spectra}
%\caption{Spectral averages and medians with the HI line placed in either the LSB or USB.}
%\label{fig:rawspectra}
%\end{figure}

\enddocument
